\documentclass[a4paper,french,towsides,10pt]{book}
\usepackage[utf8]{inputenc}
\usepackage[french]{babel}
\usepackage{fancyhdr}
\usepackage{enumerate}
\usepackage{graphicx}
\usepackage{multirow}
\usepackage{placeins}
\usepackage{tabularx}
\usepackage[french,ruled,vlined,linesnumbered]{algorithm2e}
\usepackage{amsmath}
\usepackage{amssymb}
\usepackage[bookmarks=true]{hyperref}
\hypersetup{pdfborder={0 0 0}}
\pagestyle{fancy}
\setlength{\parskip}{1.5ex plus .4ex minus .4ex}
\renewcommand{\labelitemi}{\textbullet}
\renewcommand{\chaptermark}[1]{\markboth{#1}{}}



\pagestyle{fancy}

\renewcommand{\chaptermark}[1]{\markboth{#1}{}}
\renewcommand{\sectionmark}[1]{\markright{\thesection\ #1}}

\fancyhf{}

\fancyhead[RO,LE]{\thepage}
\fancyhead[LO]{\leftmark}
\fancyhead[RE]{TP FMIN105 Algorithmique / Complexité / Calculabilité}

\fancypagestyle{corps}{ 
\fancyhead[RO,LE]{\thepage}
\fancyhead[LO]{\rightmark}
\fancyhead[RE]{\leftmark}
}

\renewcommand{\footrulewidth}{0pt} % pas de filet en bas
\fancypagestyle{plain}{ % pages de tetes de chapitre
\fancyhead{}
% supprime l’entete
\renewcommand{\headrulewidth}{0pt} % et le filet
}
\newcommand{\clearemptydoublepage}{%
	\newpage{\pagestyle{empty}\cleardoublepage}}


%Modification des marges
%\\oddsidemargin}{-2,5cm}
%\addtolength{\textwidth}{5cm}
%\addtolength{\topmargin}{-2,5cm}
%\addtolength{\textheight}{4cm}

%definition des fonctions de la page de garde
\input{includes/gardedef}

%definition du titre et autres param
\def\titre{\LARGE TP FMIN105 \\ Algorithmique / Complexité / Calculabilité}
\def\sstitre{Rapport (Décembre 2011)}
\def\auteurs{
      Thibaut \textsc{Marmin} \\
      Clément \textsc{Sipieter} \\
      William \textsc{Dyce}}

\begin{document}
\renewcommand{\labelitemii}{\textasteriskcentered}
\thispagestyle{empty}
  \vbox to .9\vsize{%
  \vss
  \vbox to 1\vsize{%
    \haut{}{\blurb}{}
    \vfill
    
    \noindent\rule{\linewidth}{.5pt}
    \ligne{\vspace{1.5mm}\titre}
    \noindent\rule{\linewidth}{.5pt}
    \ligne{\normalsize{\textsc{\sstitre}}}
    \vfill
    \ligne{%
      \begin{tabular}{l}
	\vspace{5mm}
      \end{tabular}
      \begin{tabular}{c}
      Travail préparé par : \\\\
       \auteurs
      \end{tabular}
    }
    \ligne{%
    \begin{tabular}{l}          
	\vspace{15mm}
      \end{tabular}
       \texttt{\url}
       }
  \vss
  }
}
\clearemptydoublepage
\tableofcontents
\clearemptydoublepage
\chapter{Partie théorique}

\section{Algorithmique}
\subsection{Fonction chromatique $P_G(k)$}
Le nombre de manières de colorier un graphe est le produit des nombres de façons de colorier chaque arc.
\begin{itemize}
\item Si le graphe $G$ est complet, on aura $k$ couleurs possibles pour le premier sommet, $(k-1)$ pour le deuxième, etc\ldots (Le graphe G étant complet, la couleur du premier sommet est nécessairement exclu des autres sommets).

Le $n$\^{ième} sommet pourra être colorié de $k-(n-1)$ manières. D'où :
\[ P_{K_n}(k)=\prod_{i=0}^{n-1}(k-i) \]

\item Si $G$ est vide, la coloration d'un sommet ne contraint pas la coloration des autres sommets. On obtient alors :
\[ P_{\overline{K_n}}(k)=k^n \]
\end{itemize}

\subsection{Nombre chromatique $\chi(G)$}
On l'appelle "nombre chromatique" de $G$: $\chi(G)$ étant, par définition, le nombre minimum de couleurs nécessaires pour colorier $G$, si $k < \chi(G)$ alors le graphe $G$ ne peut pas être colorié par $k$ couleurs. Si $k \geq \chi(G)$ alors il doit y avoir au moins une manière de colorier $G$, celui utilisant $\chi(G)$ couleurs.

On a donc :
\begin{displaymath}
	P_G(k) \left\{ \begin{array}{ll}
	=0 & \textrm{si $k < \chi(G)$} \\
	\geq 1 & \textrm{sinon}
	\end{array} \right.
\end{displaymath}

\subsection{Décomposition de $P_G$}
Montrons d'abord que la propriété est vraie pour tout graphe complet $K_n$. Pour commencer on remarque que, pour tout arrête $e$:

\begin{itemize}
\item $K_{n\backslash e}$ est exactement $K_{n-1}$, et donc: 

\[ P_{K_n\backslash e}(k) = P_{K_{n-1}} = \prod_{i=0}^{n-2}(k-i) \]


\item Soit $e = (a,b)$. On peut supposer (sans perte de généralité) que $b$ est considéré en dernier lors de la coloration de $K_n$, donc qu'il lui reste $k-(n-1)$ couleurs. Pour colorier $K_{n-e}$ on aura un choix de plus pour lui, à savoir la couleur de $a$, donc $k-(n-2)$ en totale. De ce fait:

\[ P_{K_n-e}(k) = P_{K_{n-1}}(k)(k-(n-2)) = (\prod_{i=0}^{n-2}(k-i))(k-(n-2)) \]
\end{itemize}

On a donc très clairement:
\begin{eqnarray*}
P_{K_n-e}(k) -  P_{K_n\backslash e}(k) &=& (\prod_{i=0}^{n-2}(k-i))(k-(n-2)) - \prod_{i=0}^{n-2}(k-i)\\
&=& \prod_{i=0}^{n-2}(k-i)(k-(n-1))\\
&=& \prod_{i=0}^{n-1}(k-i)\\
&=& P_{K_n}(k)
\end{eqnarray*}
Tout graphe de rang $n$ pouvant se générer à partir de $K_n$ (en enlevant des arrêtes) on cherchera à prouver que la suppression d'arrête conserve notre propriété. Autrement dit on aimerait montrer que pour tout graphe $G$ et tout arrête $a$ de celui-ci:

\begin{eqnarray*}
&&P_G(k) = P_{G-e}(k) - P_{G \backslash e}(k) \\
&\Rightarrow&  P_{G-a}(k) = P_{G-e-a}(k) - P_{G \backslash e-a}(k)
\end{eqnarray*}

On supposera évidemment que $a$ et $e$ sont distinctes. 

TODO FINISH

\subsection{Polynôme chromatique ?}
Soit $H$ un prédicat tel que :
\begin{displaymath}
	H(m) = \left\{ \begin{array}{ll}
	\top & \textrm{si $\forall$ $G$, graphe de $m$ arrêtes ou moins, $P_G(k)$ est polynomiale.} \\
	\bot & \textrm{sinon.}
	\end{array} \right.
\end{displaymath} 
\begin{itemize} 
\item Nous rappellons que $P_{\overline{K_n}}(k)=k^n$, donc $H(0)$ est vraie. 
\item Supposons $\exists m \in \mathbb{N}$ $|$ $H(m)$ l'est également. Ajoutons l'arc $a$ à $G$. $G_{m+e}$ est un graphe à $(m+1)$ arrêtes :

\[ P_{G_{m+z}} = P_{G_{m+e}-e} - P_{G_{m+e} \backslash e} \]

Clairement $P_{G_{m+1}-e}$ et $P_{G_{m+1} \backslash e}$ ont $(m+1)-1 = m$ arrêtes. Or par hypothèse de recurrence $H(m)$ est vraie,
$P_{G_{m+1}}$ est la différence entre deux polynomiales, donc est polynomiale lui-même. On a donc $H(m+1)$.
\item On vient de montrer $(H(0) \wedge (H(m) \Rightarrow H(m+1)))$. Par récurrence on a donc $H(m)$ vrai $\forall m \in \mathbb{N}$. 

\end{itemize} 

\subsection{Application de la décomposition}
Utilisons la formule trouvée au point précédent, et admettons que pour $P_n$ une chaîne de taille $n$ on a :
\[ P_{P_n}(k) = k(k-1)^{n-1} \]

Prenons $A$ le graphe initial :
\begin{eqnarray*}
P_A(k) & = & P_B(k) - P_C(k) \\
&=& \big(P_D(k)-P_E(k)\big)-\big(P_F(k)-P_{P_3}(k)\big)\\
&=& \Big[\big(P_{P_5}(k)-P_{P_4}(k)\big)-\big(P_{P_4}(k)-P_{P_3}(k)\big)\Big]-\Big[\big(P_{P_4}(k)-P_{K_3}(k)\big)-P_{P_3}(k)\Big]\\
&=& P_{P_5}(k)+2P_{P_3}(k)-P_{K_3}(k)+3P_{P_4}(k)\\
&=& k(k-1)^4 +2k(k-1)^2+k(k-1)(k-2)-3(k-1)^3 \\
&=& k(k-1)\big[(k-1)^3+z(k-1)+(k-z)-3(k-1)^2\big]\\
&=& (k^2-k)\Big[(k-1)^2\big((k-1)-3\big)+3k -4 \Big]\\
&=& (k^2-k)[k^3-6k^2+12k-8]\\
&=& k^5-7k^4+18k^3-20k^3+8k
\end{eqnarray*}
Où :

$A$ : \raisebox{-0.5\height}{\includegraphics[width=2cm]{files/gAex1.pdf}}

\begin{tabular}{llcll}
$B = A-(e,d) $ : & \raisebox{-0.5\height}{\includegraphics[width=2cm]{files/gBex1.pdf}} & et & $C = A\backslash(e,d)$ : & \raisebox{-0.5\height}{\includegraphics[width=2cm]{files/gCex1.pdf}} \\
$D = C-(a,b)$ : & \raisebox{-0.5\height}{\includegraphics[width=2cm]{files/gDex1.pdf}} & et & $E = C\backslash(e,b)$ : & \raisebox{-0.5\height}{\includegraphics[width=2cm]{files/gEex1.pdf}} \\
$F = C-(b,ed)$ : & \raisebox{-0.5\height}{\includegraphics[width=2cm]{files/gFex1.pdf}}& et & $C \backslash(b,ed) = P_3$ : & \raisebox{-0.5\height}{\includegraphics[width=2cm]{files/gC-ex1.pdf}}
\end{tabular}

\subsection{Particularités des polynômes chromatiques}
Nous cherchions à prouver que pour tout graphe $G$ à $n$ sommets et $m$ arrêtes, avec $C = \{c_i\}$ un ensemble de coéfficients naturelles (donc positives) : 
\[ P_G(k) = k^n - mk^{n-1} + \sum_{i = 2}^n \Big((-1)^i(c_ik^{n-i})\Big) \]
On nomera $F_k(n,m,C)$ cette forme polynomiale particulière. Soit $H$ un prédicat tel que :
\[ H(n,m) \Leftrightarrow \forall{G(n,m)}, \hspace{1mm} \exists{C} \hspace{1mm} 
										| \hspace{1mm} P_G(k) = F(n, m, C) \]
On montrera par récurance que $H(n,m)$ est vrai pour tout $n$ et tout $m$ :
\begin{itemize}
\item \textbf{Sommets -- cas de base} \\
Pour $n = 1$ on a $P_G(k) = k$, donc $H(0,1)$ est vérifié.
\item \textbf{Sommets -- pas récursif} \\ Supposons qu'il existe un nombre de sommets $n_0$ et un nombre d'arrêtes $m_0$ tel que $H(m_0,n_0)$ soit vrai. Soit $G$ un graphe à $n_0$ sommets et $m_0$ arrêtes, et $G^+$ le graphe généré en reliant un nouveau sommet $s'$ à une selection de $n' \leq n_0$ sommets de $G$. On aura donc $(k-n')$ manières de colorier $s'$. De ce fait:
\[ P_{G^+}(k) = P_G(k)(k-n') \]
Où $P_G(k)$ est de la forme $F$. $P_G^+(k)$ est donc de la forme:
\begin{eqnarray*}
&&	F(n,m,C)(k-n')			\\
&=& F(n,m,C)k - n'F(n,m,C)	
\end{eqnarray*} 
\begin{itemize}
\item \textbf{Maintient des signes alternatifs} \\ 
Le fait de multiplier par $k$ "décale" les termes du polynomiale à gauche. Soustraire $n'F_k(n,m,C)$ retranche alors au coéfficient de chaque terme $n'$ fois celui de leur voisin de gauche, avec $0 \leq n' \leq n$. Or si le term est positive, ce fameux voisin de gauche est négative par hypothèse, donc son coéfficient augmentera. De même si le term est négative son coéfficient va diminuer. Les signes resteront donc alternatifs dans $P_G^+(k)$.
\item \textbf{Maintient de $k^n$} \\
$k^{n}$ deviendra $k^{n+1}$ dans $F_k(n,m,C)k$ et, n'aillant pas de voisin gauche dans $F_k(n,m,C)$, on ne lui retranchera rien. La propriété sur le terme de plus haut degré est donc maintenu.
\item \textbf{Maintient de $mk^{n-1}$} \\
Le terme $-mk^{n-1}$, aillant pour voisin gauche $k^n$, deviendra \[-mk^n - n'k^n = -(m+n')k^n \] $(m+n')$ étant le nombre d'arrêtes de $G^+$. La propriétée sur le second terme de plus haut degré est alors maintenu aussi.
\end{itemize}

\item \textbf{Arrêtes -- cas de base} \\
Pour $m = 0$ on a $G = \overline{K}_n$ et donc $P_G(k) = P_{\overline{K}_n}(k) = k^n$. Du coup $\forall n \in \mathbb{N^+}$, $H(0,n)$ est vrai.
\item \textbf{Arrêtes -- pas récursive} \\
Supposons qu'il existe un nombre d'arrêtes $m_0$ tel que pour tout $m \leq m_0$ et tout $n$ on a $H(m_0,n)$. Soit $G$ un graphe à $m_0 + 1$ arrêtes. D'après la formule du question $3$ on a donc:
\begin{eqnarray*}
P_G(k) &=& P_{G-e}(k) - P_{G \backslash e}(k)	\\
		&=& A - B									
\end{eqnarray*}
Où $A$ et $B$ sont de la forme $F_k$. $P_G(k)$, lui, est donc de la forme:
\[ F_k(n,m-1,C') - F_k(n-1,m-1,C'') \]

\end {itemize}

\subsection{Polynôme chromatique de $K_{1,5}$}
$K_{1,5}$ étant un arbre, on aura $k$ choix de coloration pour la racine, peu importe le choix de celle-ci, et $k-1$ pour les autres, car chacun qu'on considère sera relié à exactement une autre déjà colorié. En totale ça nous fait donc:
\begin{eqnarray*}
P_{K_{1,5}}(k) 	& = & k(k-1)^5 \\
				& = & k{((k-1)^2)}^2(k-1)	\\
				& = & k(k^2 - (2k - 1))^2(k-1)	\\
				& = & (k^5 - 4k^4  + 6k^3 - 4k^2 + k)(k-1)  \\
				& = & k^6 - 5k^5  + 10k^4 - 10k^3 + 5k^2 - k	 
\end{eqnarray*}

\subsection{Coloration de graphes non-connexes}
La coloration de chaque composante connexe $C_i$ n'influx pas sur celui des autres. Du coup le nombre de manières de colorier un graphe entier est le produit des polynômes chromatiques de ses composantes connexes:
\begin{eqnarray*}
G = \bigcup_{i=0}^n C_i & \Rightarrow & P_G(k)=\prod_{i=0}^n P_{C_i}(k) \\
\end{eqnarray*}

\subsection{Coloration d'arbres}
Supposons qu'on ait un graphe $G$ tel que $P_G(k) = k(k-1)^{n-1}$ :
\begin {itemize}
\item Intuitivement ceci veut dire qu'on a $k$ choix de couleurs pour le premier sommet colorié, puis $(k-1)$ pour chacun des $(n-1)$ autres. Du coup chaque sommet, lors de sa coloration, ne doit être en contact qu'avec un seul sommet déjà colorié. Ceci n'est possible que dans un graphe sans cycle, car même avec un seul cycle on aura au plus $k(k-1)^{n-2}(k-2) < k(k-1)^{n-1}$ colorations différentes.
\item De même si $G$ est une forêt, même s'il n'est composé que de deux composantes connexes, il devraient y avoir au moins $k^2(k-1)^{n-2} > k(k-1)^{n-1}$ colorations possibles.
\end {itemize}
On a donc $G$ connexe et sans cycle: c'est la définition même d'un arbre.

\subsection{$k^5 - 4k^4 + 6k^3 - 4k^2 + k$}
Grace aux dévelopments précédentes (questions $7$ et $12$) on reconnait:
\begin{eqnarray*}
&&		k^5 - 4k^4 + 6k^3 - 4k^2 + k 	\\	
&=&		k(k-1)^4						\\
&=&		P_{P_5}(k)						
\end{eqnarray*}
Notre premier exemple sera donc $P_5$ le chemin de taille $5$. Ensuite d'après la propriété de la question $6$ nous ne cherchions que des graphes aillant $5$ sommets et $4$ arrêtes, et d'après celle de la question $9$ ils doivent en plus être arbres. Nous proposons donc le graphe étoile $S_5 = K_{1,4}$ ainsi que l'arbre à $5$ sommets dont la particularité est d'être sans particularité (on l'appellera $A_5$) :

\begin{center}
$P_5$ : \raisebox{-0.5\height}{\includegraphics[width=2cm]{files/p5ex1.pdf}} \hspace{1cm}
$S_5$ : \raisebox{-0.5\height}{\includegraphics[width=2cm]{files/s5ex1.pdf}} \hspace{1cm}
$A_5$ : \raisebox{-0.5\height}{\includegraphics[width=2cm]{files/a5ex1.pdf}}
\end{center}

\subsection{Polynôme chromatique de $K_{2,5}$}
Nous avions déjà calculé le polynôme chromatique de $K_{1,5}$ :
\[ P_{K_{1,5}}(k) = k^6 - 5k^5  + 10k^4 - 10k^3 + 5k^2 - k \]
Or $K_{2,5}$ se construit à partir de $K_{1,5}$ par l'ajout d'un sommet relié au $5$ de la partition majoritaire. Ce nouveau sommet pourra être colorié de $(k-5)$ manières, car seront interdis les couleurs de ses $5$ voisins. On a donc :
\begin{eqnarray*}
P_{K_{2,5}} & = &	(P_{K_{1,5}}(k))(k-5)								\\
			& = & 	(k^6 - 5k^5  + 10k^4 - 10k^3 + 5k^2 - k)(k-5)		\\	
			& = &	k^7 - 10k^6 + 35k^5 - 60k^4 + 55k^3 - 26k^2 + 5k	
\end{eqnarray*}

\subsection{Polynômes chromatiques de $C_4$ et $C_5$}
\begin {itemize}
\item Pour calculer $P_{C_4}$, commençons par constater que $C_3 = K_3$ :
\begin{eqnarray*}
P_{C_4}(k)	& = &	P_{P_4}(k) - P_{K_3}(k)				\\
			& = & 	k(k-1)^3 - k(k-1)(k-2)				\\
			& = & 	(k^2 - k)((k-1)^2 - (k-2))			\\
			& = &	(k^2 - k)(k^2 - 3k + 3)				\\
			& = & 	k^4 - 4k^3 + 6k^2 - 3k							
\end{eqnarray*}
\item On peut alors utiliser $P_{C_4}$ pour calculer $P_{C_5}$ :
\begin{eqnarray*}
P_{C_5}(k)	& = &	P_{P_5}(k) - P_{C_4}(k)				\\
			& = & 	k(k-1)^4 - k^4 -4k^3 + 6k^2 - 3k	\\
			& = &	k^5 - 5k^4 + 10k^3 - 10k^2 + 4k		 
\end{eqnarray*}
\end {itemize}

\subsection{Coloration de cycles}
Soit $H$ un prédicat tel que : 
\[ H(n) \Leftrightarrow \Big(P_{C_n}(k) = \big((k-1)^n + (-1)^n(k-1)\big)\Big) \]
\begin{itemize}
\item Prenons comme cas de base $n = 3$ :
\begin{eqnarray*}
P_{C_3}(k) 				& = &	P_{K_3}(k)						\\ 
						& = &	k(k-1)(k-2)						\\
						& = &	k^3 - 3k^2 + 2k					\\
						& = &   k^3 - 3k^2 + 3k - 1 - k + 1 	\\
						& = &	(k^2 -2k + 1)(k-1) - (k-1)		\\
						& = &	(k-1)^3 -(k-1) 					
\end{eqnarray*} $H(3)$ est donc vrai.
\item Supposons $H(n)$ vrai pour un $n$ donnée. Suivant cette hypothèse on a :
\begin{eqnarray*}
P_{C_{n+1}}(k)			& = & P_{C_{n+1}-e} - P_{C_{n+1}\backslash{e}}(k)	\\
						& = & P_{P_{n+1}}(k) - P_{C_n}(k)					\\
						& = & k(k-1)^n - \Big((k-1)^n + (-1)^n(k-1)\Big)	\\
						& = & (k-1)(k-1)^n - (-1)^n(k-1)					\\
						& = & (k-1)^{n+1} (-1)^{n+1}(k-1)					
\end{eqnarray*}
On a donc $H(n+1)$.
\end{itemize}
Résultat des courses : $H(3) \wedge (H(n) \Rightarrow H(n+1))$. Par récurrence on a donc $H(n)$ vrai pour tout $n \geq 3$. À noter qu'un cycle de taille moins de $3$, ça se voit pas souvent.

\subsection{Coloration de graphes bipartis complets}
Comme d'habitude, posons $H$ un prédicat tel que : 
\[ H(n) \Leftrightarrow \Big(P_{K_{2,n}}(k) = \big(k(k-1)^n + k(k-1)(k-2)^n\big)\Big) \]
\begin{itemize}
\item Prenons comme cas de base $n = 1$. Or $K_{2,1} = P_3$ donc :
\begin{eqnarray*}
P_{K_{2,1}}(k)			& = & P_{P_3}(k)				\\
						& = & k(k-1)^2					\\
						& = & k(k-1)(1 + (k-2))			\\
						& = & k(k-1)^1 + k(k-1)(k-2)^1	
\end{eqnarray*}
$H(1)$ est donc vrai.
\item On appelle $C_{2,n}$ le graphe pseudo-biparti complet à $n$ sommets, qui se génère à partie de $K_{2,n}$ en reliant les deux sommets de la première partition :

\begin{center}
$C_{2,n}$ : \raisebox{-0.5\height}{\includegraphics[width=2cm]{files/c2nex1.pdf}}
\end{center}

On appelle cette arrête spéciale $e^*$. On a donc :
\[ C_{2,n}-e^* = K_{2,n} \]
et
\[ C_{2,n} \backslash{e^*} = S_n \]
On rappelle que $S_n$ est un arbre. Finalement on admettera :
\[ \forall{e} \hspace{3mm} K_{2,n}-e = K_{2,n}(k-1) \]
En effet il ne reste que $K_{2,n-1}$ plus un sommet pendant, donc coloriable de $k-1$ manières :

\begin{center}
$K_{2,n-1}^{+1}$ : \raisebox{-0.5\height}{\includegraphics[width=2cm]{files/k2n_dangle_ex1.pdf}}
\end{center}

Supposons alors $H(n)$ vrai pour un $n$ donnée. 
\begin{eqnarray*}
P_{K_{2,n+1}}(k)			& = & P_{K_{2,n+1}-e} - P_{K_{2,n+1} \backslash{e}}				\\
							& = & P_{K_{2,n}}(k-1) - P_{C_{2,n}}							\\
							& = & P_{K_{2,n}}(k-1) - P_{K_{2,n}} + P_{S_n}					\\
							& = & (k-2)\big(k(k-1)^n + k(k-1)(k-2)^n\big) + k(k-1)^{n-1}	\\
							& = & k(k-1)^n(k-2) + k(k-1)(k-2)^{n+1} + k(k-1)^{n-1}			\\
							& = & k(k-1)^{n-1}\big((k-1)(k-2) + 1\big) + k(k-1)(k-2)^{n+1}	\\
							& = & FIXME														\\
							& = & k(k-1)^{n+1} + k(k-1)(k-2)^{n+1}
\end{eqnarray*}
On a donc $H(n+1)$.
\end{itemize}
Comme avant $H(1) \wedge (H(n) \Rightarrow H(n+1))$ implique par récurrence que $H(n)$ sera vrai pour tout $n \geq 1$.

\newpage
\section{Complexité}
\subsection{SAT $\propto$ 3--SAT}
\begin{enumerate}[(a)]
\item \begin{description}
\item[Énoncé de SAT] : \\
\begin{tabular}{r l l}
Données : & $ \mathcal{V} = \lbrace v_1, v_2 \ldots v_n \rbrace $ & \emph{Ensemble de $n$ variables}\\
& $ \mathcal{C} = \lbrace c_1, c_2, c_3 \ldots c_m \rbrace $ & \emph{Ensemble de $m$ clauses}\\
où & $ c_i = ( l_{i1} \vee l_{i2} \vee \cdots \vee l_{ik} ) $ & \emph{Clauses de $k$ littéraux}\\
avec & $ l_{ij} = v$ ou $ \neg v $ & \emph{avec $v \in U$} \\
\end{tabular}

Problème : existe-il au moins une affectation des variables telle que chaque clause de $\mathcal{C}$ soit vrai.

\item [Énoncé de 3--SAT] : 

3--SAT est identique au problème SAT avec $k = 3$.\\
\begin{tabular}{r l}
Données : & $ \mathcal{V} = \lbrace v_1, v_2, v_3 \ldots v_n \rbrace $\\
& $ \mathcal{C} = \lbrace c_1, c_2, c_3 \ldots c_m \rbrace $\\
où & $ c_i = ( l_{i1} \vee l_{i2} \vee l_{i3} ) $\\
avec & $ l_{ij} = v$ ou $ \neg v$\\
\end{tabular}
\end{description}
\item La réduction du problème SAT peut être définit en montrant que chaque clause $c$ de $\mathcal{C}$ peut-être transformée en un ensemble de clauses $\mathcal{C'}$ tel que pour toute affectation rendant vrai l'ensemble des clauses de $\mathcal{C}$, on peut trouver une affectation rendant vrai chaque clause de $\mathcal{C'}$. Chaque clause de $\mathcal{C'}$ devant être de taille exactement 3. La réciproque doit également être montrée.

Définissons les réductions :

\begin{description}
\item \mathversion{bold} $k = 1$ \mathversion{normal}

Soit $ci_1$ une clause de taille 1, on a $ci_1 = (l)$.
Ajoutons deux variables $v_1, v_2 \notin \mathcal{V}$ et transformons la clause $c$ en quatre clauses. On obtient l'ensemble $\mathcal{C}_1 = \lbrace c_1, c_2, c_3, c_4 \rbrace$ avec :
\[ c_1 = ( l \vee v_1 \vee v_2 )\]
\[ c_2 = ( l \vee v_1 \vee \neg v_2 )\]
\[ c_3 = ( l \vee \neg v_1 \vee v_2 )\]
\[ c_4 = ( l \vee \neg v_1 \vee \neg v_2 )\]

\item \mathversion{bold} $k = 2$ \mathversion{normal}

Soit $ci_2$ une clause de taille 2, on a $ci_2 = ( l_1 \vee l_2 )$.
Ajoutons une variable $v \notin \mathcal{V}$ et transformons la clause $c$ en deux clauses. On obtient l'ensemble $\mathcal{C}_2 = \lbrace c_1, c_2 \rbrace$ avec :
\[ c_1 = ( l_1 \vee l_2 \vee v )\]
\[ c_2 = ( l_1 \vee l_2 \vee \neg v )\]

\item \mathversion{bold} $k = 3$ \mathversion{normal}

La clause $ci_3$ ne subit pas de transformation.
\[ \mathcal{C}_3 = \lbrace ci_3 \rbrace \]

\item \mathversion{bold} $k > 3$ \mathversion{normal}

Soit la clause $ci_k = ( l_1 \vee l_2 \vee \cdots \vee l_k )$. On ajoute $(k - 3)$ nouvelles variables $(v_1, v_2 \ldots v_{k-3})$.

\[ \mathcal{C}_k = \underbrace{(l_1 \vee l_2 \vee v_1)}_{c_1} \bigwedge_{i=1}^{k-4}\left[ \underbrace{(\neg v_i \vee l_{i+2} \vee v_{i+1})}_{c_{i+1}}\right]  \wedge \underbrace{(\neg v_{k-3} \vee l_{k-1} \vee l_{k})}_{c_{k-2}} \]

Montrons que SAT est vrai si et seulement si 3--SAT est vrai :

\begin{description}
 \item \textbf{SAT $\rightarrow$ 3--SAT}
 
 	\begin{itemize}
 		\item Soit une interprétation $I_1$ qui satisfasse la clause $ci_1$ :
 		 \[ val(I_1,ci_1) = val(I_1,l) = vrai\]
 		
 		Prenons une interprétation $I_1'$ avec $val(I_1,l) = val(I_1',l)$, peu importe les affectations de $v_1$ et $v_2$, $l$ étant présent dans toutes les clauses de $\mathcal{C}_1$ :
 		\[ val(I_1',\mathcal{C}) = vrai \]
 		
 		\item Soit une interprétation $I_2$ qui satisfasse la clause $ci_2$ :
 		\[ \exists i, val(I_2,l_i) = vrai \]
 		Prenons une interprétation $I_2'$ avec :
 		\[ val(I_2,l_1) = val(I_2',l_1) \]
  		\[ val(I_2,l_2) = val(I_2',l_2) \]
  		Peu importe l'affectation de $v$ dans $I_2'$, on a $val(I_2',\mathcal{C}_2) = vrai$.
  		
  		\item Soit une interprétation $I_k$ qui satisfasse la clause $ci_k$ :
  		\[ \exists i, val(I_k,l_i) = vrai \]
  		
  		Prenons une interprétation $I_k'$ telle que :
  		\begin{eqnarray*}
  		val(I_k,l_i) & = & val(I_k',l_i)  \\
  		\forall j \in [1;(i-2)], val(I_k', v_j) & = & vrai \\
  		\forall j \in [(i-1);(k-3)], val(I_k', v_j) & = & faux \\
  		\end{eqnarray*}
  		
  		On obtient :
  		\[ val(I_k',\mathcal{C}_k) = vrai\]
  		
 	\end{itemize}
 	
 \item \textbf{3--SAT $\rightarrow$ SAT}
 
 	\begin{itemize}
 	\item Prenons une interprétation $I_1$ telle que $val(I_1,\mathcal{C}_1) = vrai$.
 	
 	Sans perte de généralité, on suppose que : 
 	\[ val(I_1,v_1) = val(I_1,v_2) = vrai \]
 	La clause $c_4$ de $\mathcal{C}_1$ ne peut être satisfaite que si $val(I_1,l) = vrai$.
 	
 	On a donc :
 	\[ val(I_1,ci_1) = vrai \]
 	
 	\item Prenons une interprétation $I_2$ telle que $val(I_2,\mathcal{C}_2) = vrai$.
 	
 	Sans perte de généralité on suppose que :
 	\[val(I_2,v) = vrai \]
 	
 	La clause $c_2$ de $\mathcal{C}_2$ ne peut être satisfaire que si $val(I_2,(l_1 \vee l_2)) = vrai$.
 	
 	On a donc :
 	\[ val(I_2,ci_2) = vrai \]
 	
 	\item Prenons une interprétation $I_k$ telle que $val(I_k,\mathcal{C}_k) = vrai$ et montrons qu'il existe forcément un $i$ tel que $val(I_k,l_i) = vrai$.
 	
 	Supposons que l'interprétation $I_k$ est modèle de $\mathcal{C}_k$ avec 
 	\[ \forall i \in [1;k], val(I_k,l_i) = faux \]
	\[ \Rightarrow val(I_k,v_1) = vrai  \textrm{ (dans $c_1$)} \]
	Donc :
	
	\begin{tabular}{lrl}
		& $\forall i \in [1;(k-4)], val(I_k,v_{i+1})$ & $= vrai$ \\
		$\Rightarrow$ & $val(I_k,v_{k-3})$ & $= vrai$ \\
		$\Rightarrow$ & $val(I_k,c_{k-2})$ & $= faux$ \\
		$\Rightarrow$ & $val(I_k,\mathcal{C}_k)$ & $= faux$
	\end{tabular}
	
	Pour que l'interprétation $I_k$ satisfasse $\mathcal{C}_k$, il doit exister un $i \in [1;k]$ tel que $val(I_k,l_i) = vrai$.
	
	On a donc :
	\[ val(I_k,ci_k) = vrai \]
 	 	
 	\end{itemize}
 
\end{description}

\end{description}

\item Le point (b) définit la réduction de SAT vers 3--SAT. Afin de montrer la NP-Complétude de 3--SAT, montrons que la réduction s’effectue en un temps polynomial.

Soit :
\begin{description}
\item $k$ la taille de la clause initiale,
\item $v_k$ le nombre de variables à ajouter pour obtenir des clauses de taille 3,
\item $w_k$ le nombre de clauses de taille 3 obtenues à partir de la clause initiale.
\end{description}
\begin{center}
\begin{tabular}{c c}
$v_3 = 0$ & $w_3 = 1$ \\
$v_4 = 1$ & $w_4 = 2$ \\
$v_5 = 2$ & $w_5 = 3$ \\
\vdots & \vdots
\end{tabular}
\end{center}

Pour tout $k > 3$ :
\[ v_k = v_{\left \lceil \frac{k}{2} \right \rceil + 1} + v_{\left \lfloor \frac{k}{2} \right \rfloor + 1} + 1 \]
\[ w_k = w_{\left \lceil \frac{k}{2} \right \rceil + 1} + w_{\left \lfloor \frac{k}{2} \right \rfloor + 1} \]

$v_k = \theta(k)$, donc borné par la taille de F. La réduction s'effectue donc en un temps polynomial.

Il est possible de réduire le problème SAT à 3--SAT en un temps polynomial, SAT étant NP-complet, 3--SAT l'est aussi.

\item 
\end{enumerate}

\subsection{NP-complétude de 2--SAT}
\begin{enumerate}
\item coucou
\item coucou
\end{enumerate}

\subsection{2--SAT, un problème polynomial}
\begin{enumerate}
\item coucou
\item coucou
\end{enumerate}

\newpage
\section{Calculabilité}
\begin{enumerate}
\item La stratégie d'énumération des couples d'entier peut être visualisée sur un graphique en suivant les diagonales successives comme sur l'image\footnote{Image provenant de Wikipedia, ce fichier est disponible selon les termes de la licence
 Creative Commons.} suivante :
\begin{figure*}[!h]
\begin{center}
\includegraphics[width=0.6\textwidth]{files/diagcouples.pdf}
\caption{La fonction de couplage de Cantor établit une bijection de $\mathbb{N}*\mathbb{N}$ dans
 $\mathbb{N}$.}
 \end{center}
\end{figure*}

Soit $(x,y) \in \mathbb{N}*\mathbb{N}$ un couple. On trie par ordre lexicographique $(x+y)$. Ainsi on obtient le tableau suivant :

\begin{tabularx}{1.1\textwidth}{| m{1.5cm} | X | X | X | X | X | X | X | X | X | X | X }
\hline
$(x,y)$ & $(0,0)$ & $(1,0)$ & $(0,1)$ & $(2,0)$ & $(1,1)$ & $(0,2)$ & $(3,0)$ & $(2,1)$ & $(1,2)$ & $(0,3)$ & \ldots \\
\hline
$(x+y)$ & 0 & 1 & 1 & 2 & 2 & 2 & 3 & 3 & 3 & 3 & \ldots \\
\hline
$c_2(x+y)$ & 0 & 1 & 2 & 3 & 4 & 5 & 6 & 7 & 8 & 9 & \ldots \\
\hline
\end{tabularx}

\item
\begin{description}
\item[Fonction de codage]
\[ c_2(x,y)= \frac{(x+y)(x+y+1)}{2}+y \]

\item[Fonctions de décodage]
Les fonctions de décodage ne peuvent pas être décrites sous la forme de formules arithmétiques. Elles nécessitent l'algorithme suivant :

\begin{algorithm}[H]
  \caption{CalculXY($z$)}
  \Donnees{
  $z$ \textit{// Rang du couple (x,y)}
  }
  \Deb{
  $s \leftarrow 0$\;
  $t \leftarrow 0$\;
    \Tq{$s \leqslant z$}{
		$s \leftarrow \frac{t*(t+1)}{2}$\;
		$t \leftarrow t+1$\;
    }
    $t \leftarrow t-2$\;
    $z \leftarrow \frac{t*(t+1)}{2}$\;
    $y \leftarrow z-s$\;
    $x \leftarrow t-y$\;
    \Retour Couple($x$,$y$)\;
  }
\end{algorithm}
\end{description}

\item 
\begin{description}
\item[Codage des triplets] : il peut avoir lieu de manière récursive :
\[ c_3(x,y,z)=c_2(x,c_2(y,z)) \]
\item[Généralisation au codage des k-uplets] : 
\begin{eqnarray*}
& &c_k(x_1,x_2,\ldots,x_k)=c_2(x_1,c_{k-1}(x_2,\ldots,x_k)) \\
\textrm{Avec : } & &c_2(x,y)=\frac{(x+y)(x+y+1)}{2}+y
\end{eqnarray*}



\end{description}

\item Prenons une suite $r=(r_1,r_2,r_3,\ldots)$ qui énumère les réels de l'intervalle $[0;1]$, puis créons un réel x compris dans cet intervalle, tel que si la n\up{ième} de $r_n$ est égale à $1$, la n\up{ième} décimale de $x$ est égale à 2. Dans la cas contraire, la n\up{ième} décimale de $x$ est égale à 1.

On obtient sur cet exemple :

\begin{tabular}{r c c c c c c c c c c c c}
$r_1$ & = &0&,&\textbf{4}&2&9&6&4&6&1 &\ldots\\
$r_2$ & = &0&,&2&\textbf{7}&3&2&9&4&0 &\ldots\\
$r_3$ & = &0&,&6&4&\textbf{1}&1&5&1&2 &\ldots\\
$r_4$ & = &0&,&3&0&5&\textbf{9}&0&4&3 &\ldots\\
$r_5$ & = &0&,&9&1&3&3&\textbf{1}&8&2 &\ldots\\
$r_6$ & = &0&,&0&2&0&8&3&\textbf{2}&7 &\ldots\\
$r_7$ & = &0&,&2&5&7&3&6&4&\textbf{0} &\ldots\\
\vdots & \vdots & \vdots & \vdots & \vdots & \vdots & \vdots & \vdots & \vdots & \vdots & \vdots & $\ddots$ \\
&&&&$\downarrow$ &$\downarrow$ &$\downarrow$ &$\downarrow$ &$\downarrow$ &$\downarrow$ &$\downarrow$ &\\
$x$ & = &0&,&1&1&2&1&2&1&1&\ldots\\
\end{tabular}

Le réel $x$ ne peut pas être énuméré par la suite $r$ car il diffère de sa première décimale dans $r_1$, de sa deuxième décimale dans $r_2$, \ldots de sa n\up{ième} décimale dans $r_n$. Pourtant le réel $x$ est clairement dans l'intervalle $[0;1]$.

L'ensemble des éléments de l'intervalle $[0;1]$ ne sont donc pas dénombrables, donc pas énumérables. On ne peut donc pas trouver de fonction de codage pour cet ensemble.

On peu généraliser à l'ensemble $\mathbb{R}$ : $[0;1]$ étant inclus dans $\mathbb{R}$, et $[0;1]$ n'étant pas dénombrable, l'ensemble $\mathbb{R}$ n'est pas dénombrable.

\end{enumerate}
\clearemptydoublepage
\chapter{Partie pratique}
\textit{Le but de ce TP est d'implémenter deux algorithme de résolution du problème de flot maximum : l'algorithme d'Edmonds-Karp et l'algorithme de Dinic. Nous commencerons par spécifier les fonctionnalités que devra implémenter notre programme, puis nous détaillerons la manière dont ces fonctionnalités ont été développées. Une troisième partie sera consacrée aux tests effectués sur les deux algorithmes ainsi qu'à l'analyse des résultats.}

\section{Spécification fonctionnelles}

\subsection{Résolution du problème de flot maximum}

Le programme doit être capable de :
\begin{itemize}
\item générer et d'actualiser les graphes d'écarts successifs
\item calculer la valeur du flot obtenu à partir du graphe d'écart final
\item résoudre le problème de flot maximum en suivant l'algorithme d'\textbf{Edmonds-Karp}
\item résoudre le problème de flot maximum en suivant l'algorithme de \textbf{Dinic}
\item retourner la solution de manière exploitable pour l'analyse
\end{itemize}

Il faudra veiller à conserver les complexités des deux algorithmes, notamment en prenant garde aux structures de données et librairies utilisées.

\subsection{Génération aléatoire d'un réseau de transport}

La génération aléatoire de graphes de type réseau de transport permettra de tester les deux algorithmes. Il faudra veiller à ce que le graphe respecte les conditions d'un réseau de transport notamment la possession d'une source et d'un puits, la pondération des arcs (capacités), et assurer la connexité du graphe. La génération de ce réseau de transport devra être paramétrable selon la taille (nombre de sommets) et la couverture (nombre d'arcs).

\section{Spécification technique}

\subsection{Programmation C++}
Parce qu'il s'agit d'un bon compromis entre langage orienté objet et langage de bas niveau, nous avons choisi de développer cette application en C++. Nous pourrons ainsi abstraire la gestion des graphes (notamment des structures de données) dans nos algorithmes, tout en gardant la possibilité d'optimiser le code grâce à la flexibilité du langage C.

\subsection{Structures de données}

Plusieurs structures de données sont possibles. Nous avons choisi d'implémenter une représentation par listes d'adjacences et une autre par matrice d'adjacences.
\begin{description}
\item[Listes d'adjacences] : chaque sommet possède la liste de ses voisins. Ces listes ont l'avantage d'allouer de la mémoire uniquement lorsqu'une information doit être stockée. 
\item[Matrice d'adjacences] : la mémoire allouée pour cette structure de données ne dépend que du nombre de sommets ($n^2$). Cette structure a l'avantage d'offrir un accès direct à un arc pour deux sommets donnés.
\end{description}

Dans un but d'optimisation mémoire, on utilise en général des listes d'adjacences lorsque l'on travail sur des graphes peu denses. En effet, la taille allouée par cette structure de données étant directement dépendante du nombre d'arcs, elle est donc réduite par rapport aux matrices d'adjacences. Sur des graphes très dense, on utilisera plutôt des matrice d'adjacences, permettant un accès aux données plus rapide. 

Ce choix s'effectue en général en fonction des ressources matériels disponibles.

\subsection{Modélisation}
Dans un but d'abstraction de la structure de donnée, nous avons choisi de créer une classe abstraite \texttt{AbstractGraph} dont deux classes fille héritent. Un graphe peut donc être de type \texttt{AdjacencyListGraph} ou \texttt{MatrixGraph}. Cela permet une grande généricité des algorithme développés, ce qui permet d'utiliser une structure de données de manière totalement détachée des algorithmes.
\begin{figure}[t]
\begin{center}
\includegraphics[width=\textwidth]{files/diag_class}
\end{center}
\caption{Diagramme de classes.}
\end{figure}

\FloatBarrier
\subsection{Implémentation}

\subsubsection{Types et structures}

Déclaration des types \texttt{weight\_t}, \texttt{vertex\_t} et \texttt{path\_t}, et des structures \texttt{edge}, \texttt{neighbor\_t}

\lstinputlisting[language=C++,morekeywords={}]{./sources/structs}

\subsubsection{Classe \texttt{AbstractGraph}}

Cette classe déclare une série de méthodes qui doivent êtres implémentés dans les classes filles.

Header de la classe \texttt{AbstractGraph}

\lstinputlisting[language=C++,morekeywords={}]{./sources/AbstractGraph}

\subsubsection{Classe \texttt{AdjacencyListGraph}}
Cette classe représente un réseau de transport sous la forme
de deux listes d'adjacences : une représente les 
successeurs d'un sommet, l'autre les prédécesseurs.

Ce doublon d'information permet d'accélérer l'accès aux voisins d'un sommet, notamment à ces prédécesseurs. En effet, cette méthode nous permet d'accéder aux prédécesseurs directement (complexité de $O(m)$) alors que l'accès via la liste des successeurs implique une recherche des arcs pour chaque sommet (complexité de $O(nm)$).

Ces doubles listes d'adjacences nous assurent un gain de performances en terme de rapidité, qui se fait au détriment de la quantité de mémoire utilisé, qui se trouve doublée.

Header de la classe \texttt{AdjacencyListGraph}

\lstinputlisting[language=C++,morekeywords={}]{./sources/AdjacencyListGraph}


\subsubsection{Classe \texttt{MatrixGraph}}

Header de la classe \texttt{MatrixGraph}

\lstinputlisting[language=C++,morekeywords={}]{./sources/MatrixGraph}


\section{Tests \& résultats}
bla
\subsection{Méthode de test}

\subsection{Analyse des résultats}

///////////////////////////////////////////////////////////////////////////////////////


\subsection{Génération aléatoire d'un graphe}

\begin{verbatim}
/**
 * A random flow network generator
 * Attention si le graph passé en paramètre contient des arcs ceux-ci seront
 * supprimé.
 * @param graph une référence vers un graph initialiser avec un nombre de sommets
 * @param rate la proportion d'arcs à ajouter au graphe en pourcentage par rapport au graphe complet.
 * @param min_weight valuation minimal des arcs
 * @param max_weight valuation maximal des arcs
 */
void
flowNetworkGenerator(AbstractGraph& graph, float rate, uint min_weight = 1,
    uint max_weight = 1);

\subsection{Fonctions générales}

/**
 * Cette procédure génère une chaîne de caractères représentant l'affichage
 * de la valeur total du flot sur le réseau de transport ainsi que la valeur
 * du flot sur chaque arc.
 * @param flow_network le réseau de transport
 * @param residual_network le graphe d'écart associé
 */
string
flowToString(const AbstractGraph& flow_network,
    const AbstractGraph& residual_network);



\subsection{Edmonds-Karp}

/**
 * Cette fonction retourne le plus court chemin en nombre d'arcs depuis
 * le sommet start jusqu'au sommet end
 * @param g un graphe
 * @param start le sommet de départ
 * @param end le sommet d'arriver
 * @return le plus court chemin en nombre d'arcs de start à end
 */
path_t
leastArcsPath(AbstractGraph &g, vertex_t start, vertex_t end);

/**
 * Cette fonction retourne la plus petite valuation présente sur un chemin 
 * donné dans un graphe
 * @param g un graphe
 * @param path une chemin dans g
 * @return la plus petite valuation présente sur le chemin path dans g 
 */
weight_t
lightestArc(AbstractGraph& g, path_t path);

/**
 * Cette fonction converti un chemin en chaîne de caractère dans un but d'affichage
 * @param path le chemin
 * @param g le graphe
 */
string
pathToString(path_t path, const AbstractGraph& g);

/**
 * Mise à jour du graphe d'écart depuis un chemin et la valeur du flot à ajouter
 * sur ce chemin
 * @param le graphe de couche
 * @param p le chemin
 * @param k la valeur du flot à ajouter
 */
void
updateResidualNetwork(AbstractGraph& residualNetwork, path_t p, uint k);


/**
 * algorithme d'Edmonds-Karp
 * @param flow_network le réseau de transport
 * @param src le sommet source
 * @param dest le puit
 * @return le graphe d'écart final
 */
AdjacencyListGraph
edmondsKarp(const AbstractGraph& flow_network, vertex_t src, vertex_t dest);

\end{verbatim}

\subsection{Dinic}

\begin{verbatim}
/**
 * Mise à jour du graphe d'écart depuis un flot
 * @param residual_network le graphe de couche
 * @param p le flot
 */
void
updateResidualNetwork(AbstractGraph& residual_network, AbstractGraph& flow);

/**
 * Génération du graphe de couche associé au réseau de transport
 * @param residual_network le graphe d'écart
 * @param src la source
 * @param dest le puit
 * @return le graphe de couche
 */
LevelGraph
generateLevelGraph(const AbstractGraph& residual_network, vertex_t src,
    vertex_t dest);

/**
 * Calcul du flot bloquant
 * @param level_graph le graphe de couche
 * @param src la source
 * @param dest le puit
 * @return un flot bloquant
 */
AdjacencyListGraph
blockingFlow(LevelGraph& level_graph, vertex_t src, vertex_t dest);


/**
 * algorithme de Dinic
 * @param flow_network le réseau de transport
 * @param src le sommet source
 * @param dest le puit
 * @return le graphe d'écart final
 */
AdjacencyListGraph
dinic(const AbstractGraph& graph, vertex_t src, vertex_t dest);

\end{verbatim}



\end{document}