\section{Objectifs}
Le but de ce projet est de réaliser un solveur du problème de flot maximum. Pour cela, il 
nous est proposé d’implémenter l'algorithme d'Edmonds-Karp ainsi que l'algorithme de Dinic.

\section{Spécification fonctionnelles}

Le programme devra être capable de générer aléatoirement des graphes de type \emph{réseau de transport}
comportant un nombre de sommets et d'arêtes défini. Il devra aussi proposer deux fonctions 
de résolution du problème de flots sur un tel réseau, l'une implémentant l'algorithme d'Edmonds-Karp et
l'autre celui de Dinic. Ces deux fonctions devront générer et retourner le graphe d'écart final. Enfin
le programme devra disposer d'une fonction d'affichage de la valeur du flot maximum ainsi que de la 
valeur du flot sur chaque arc.

\section{Spécification technique}

\subsection{Programmation C++}
Le programme sera réaliser en C++ afin de bénéficier des performances en termes de rapidité d'execution 
du langage ainsi que de l'aspect objet permettant d'abstraire la représentation d'une graphe.

\subsection{Structures de données}

\begin{verbatim}
typedef int weight_t;
typedef uint vertex_t;

typedef struct
{
  vertex_t u;
  vertex_t v;
} edge;

typedef struct
{
  vertex_t vertex_src;
  vertex_t vertex_dest;
  weight_t weight;
} arc_t;

typedef struct
{
  vertex_t vertex;
  weight_t weight;
} neighbor_t;
  
typedef list<vertex_t> path_t;

\end{verbatim}

\subsection{XXXXReprésentation d'un graphe}

Nous avons choisi d'abstraire la représentation d'un graphe via une
classe abstraite que nous avons nommé AbstractGraph dont nous proposerons
deux implémentations une par liste d'adjacence \emph{AdjacencyListGraph} 
et une par matrice \emph{MatrixGraph}

\begin{verbatim}
class AbstractGraph
{
public:

  AbstractGraph();
  virtual
  ~AbstractGraph() = 0;

  virtual bool
  addArc(const arc_t &arc) = 0;

  virtual bool
  addArc(vertex_t src, vertex_t dest, weight_t w) = 0;

  virtual void
  rmArc(const arc_t &arc) = 0;

  virtual void
  rmArc(vertex_t src, vertex_t dest) = 0;

  virtual void
  rmAllArc() = 0;

  virtual void
  updateArc(const arc_t &arc) = 0;

  virtual weight_t
  increaseWeight(vertex_t src, vertex_t dest, weight_t w) = 0;

  virtual uint
  getNbrVertices() const = 0;

  virtual string
  toString() const;

  virtual list<neighbor_t>
  getSuccessors(vertex_t vertex) const = 0;

  virtual list<neighbor_t>
  getPredecessors(vertex_t vertex) const = 0;

  virtual weight_t
  getWeight(vertex_t src, vertex_t dest) const = 0;

};

\end{verbatim}

\subsection{AdjacencyListGraph}

La classes AdjacencyListGraph devra représenter un graphe sous forme
de deux liste d'adjacence une liste d'adjacence représentant les 
successeur d'un sommet et l'autre représentant les prédécesseur d'un sommet.

Ce doublon de l'information permet d'accéléré l'accès au voisin d'un sommet, 
notament à ces prédécesseur. En effet en sauvegardant une liste des prédécesseurs
de chaque arcs nous accédons a un prédécesseur d'un sommet en O(m) alors 
que si nous avions qu'une liste des successeur cette accès ce ferait en 
O(nm).

Ce gain de performance en terme de rapidité ce fait au détriment de la
quantité de mémoire utilisé, mais celle-ci n'augmente que d'une valeur
constante par rapport au nombre d'arcs.

\begin{verbatim}
class AdjacencyListGraph : public AbstractGraph
{

public:
  //************************************************************************************************
  //      CONSTRUCTOR
  //************************************************************************************************

  AdjacencyListGraph(uint nbr_vertices);

  AdjacencyListGraph(const AbstractGraph& graph);
  AdjacencyListGraph(const AdjacencyListGraph& graph);

  AdjacencyListGraph &
  operator=(const AbstractGraph& graph);
  AdjacencyListGraph &
  operator=(const AdjacencyListGraph& graph);

  virtual
  ~AdjacencyListGraph();

  virtual bool
  addArc(const arc_t &arc);

  virtual bool
  addArc(vertex_t src, vertex_t dest, weight_t w);

  virtual void
  rmArc(const arc_t &arc);

  virtual void
  rmArc(vertex_t src, vertex_t dest);

  virtual void
  rmAllArc();

  virtual void
  updateArc(const arc_t &arc);

  virtual weight_t
  increaseWeight(vertex_t src, vertex_t dest, weight_t w);

  virtual uint
  getNbrVertices() const;

  virtual list<neighbor_t>
  getSuccessors(vertex_t vertex) const;

  virtual list<neighbor_t>
  getPredecessors(vertex_t vertex) const;

  virtual weight_t
  getWeight(vertex_t src, vertex_t dest) const;

private:
  list<neighbor_t> *successors, *predecessors;
  uint nbr_vertices;

protected:
  void
  _clear();

  void
  _construct(const AbstractGraph& graph);

};
\end{verbatim}

\subsection{MatrixGraph}

\begin{verbatim}
class MatrixGraph : public AbstractGraph
{

public:
  //************************************************************************************************
  //      CONSTRUCTOR
  //************************************************************************************************

  MatrixGraph(uint nbr_vertices);

  MatrixGraph(const AbstractGraph& graph);

  MatrixGraph(const MatrixGraph& graph);

  MatrixGraph &
  operator=(const AbstractGraph& graph);

  MatrixGraph &
  operator=(const MatrixGraph& graph);

  virtual
  ~MatrixGraph();

  virtual bool
  addArc(const arc_t &arc);

  virtual bool
  addArc(vertex_t src, vertex_t dest, weight_t w);

  virtual void
  rmArc(const arc_t &arc);

  virtual void
  rmArc(vertex_t src, vertex_t dest);

  virtual void
  rmAllArc();

  virtual void
  updateArc(const arc_t &arc);

  virtual weight_t
  increaseWeight(vertex_t src, vertex_t dest, weight_t w);

  virtual uint
  getNbrVertices() const;

  virtual list<neighbor_t>
  getSuccessors(vertex_t vertex) const;

  virtual list<neighbor_t>
  getPredecessors(vertex_t vertex) const;

  virtual weight_t
  getWeight(vertex_t src, vertex_t dest) const;

private:
  weight_t **matrix;
  uint nbr_vertices;

protected:
  void
  _clear();

  void
  _construct(int nbr_vertices);

  void
  _construct(const AbstractGraph& graph);

};
\end{verbatim}

\subsection{Génération aléatoire d'un graphe}

\begin{verbatim}
/**
 * A random flow network generator
 * Attention si le graph passé en paramètre contient des arcs ceux-ci seront
 * supprimé.
 * @param graph une référence vers un graph initialiser avec un nombre de sommets
 * @param rate la proportion d'arcs à ajouter au graphe en pourcentage par rapport au graphe complet.
 * @param min_weight valuation minimal des arcs
 * @param max_weight valuation maximal des arcs
 */
void
flowNetworkGenerator(AbstractGraph& graph, float rate, uint min_weight = 1,
    uint max_weight = 1);

\subsection{Fonctions générales}

/**
 * Cette procédure génère une chaîne de caractères représentant l'affichage
 * de la valeur total du flot sur le réseau de transport ainsi que la valeur
 * du flot sur chaque arc.
 * @param flow_network le réseau de transport
 * @param residual_network le graphe d'écart associé
 */
string
flowToString(const AbstractGraph& flow_network,
    const AbstractGraph& residual_network);



\subsection{Edmonds-Karp}

/**
 * Cette fonction retourne le plus court chemin en nombre d'arcs depuis
 * le sommet start jusqu'au sommet end
 * @param g un graphe
 * @param start le sommet de départ
 * @param end le sommet d'arriver
 * @return le plus court chemin en nombre d'arcs de start à end
 */
path_t
leastArcsPath(AbstractGraph &g, vertex_t start, vertex_t end);

/**
 * Cette fonction retourne la plus petite valuation présente sur un chemin 
 * donné dans un graphe
 * @param g un graphe
 * @param path une chemin dans g
 * @return la plus petite valuation présente sur le chemin path dans g 
 */
weight_t
lightestArc(AbstractGraph& g, path_t path);

/**
 * Cette fonction converti un chemin en chaîne de caractère dans un but d'affichage
 * @param path le chemin
 * @param g le graphe
 */
string
pathToString(path_t path, const AbstractGraph& g);

/**
 * Mise à jour du graphe d'écart depuis un chemin et la valeur du flot à ajouter
 * sur ce chemin
 * @param le graphe de couche
 * @param p le chemin
 * @param k la valeur du flot à ajouter
 */
void
updateResidualNetwork(AbstractGraph& residualNetwork, path_t p, uint k);


/**
 * algorithme d'Edmonds-Karp
 * @param flow_network le réseau de transport
 * @param src le sommet source
 * @param dest le puit
 * @return le graphe d'écart final
 */
AdjacencyListGraph
edmondsKarp(const AbstractGraph& flow_network, vertex_t src, vertex_t dest);

\end{verbatim}

\subsection{Dinic}

\begin{verbatim}
/**
 * Mise à jour du graphe d'écart depuis un flot
 * @param residual_network le graphe de couche
 * @param p le flot
 */
void
updateResidualNetwork(AbstractGraph& residual_network, AbstractGraph& flow);

/**
 * Génération du graphe de couche associé au réseau de transport
 * @param residual_network le graphe d'écart
 * @param src la source
 * @param dest le puit
 * @return le graphe de couche
 */
LevelGraph
generateLevelGraph(const AbstractGraph& residual_network, vertex_t src,
    vertex_t dest);

/**
 * Calcul du flot bloquant
 * @param level_graph le graphe de couche
 * @param src la source
 * @param dest le puit
 * @return un flot bloquant
 */
AdjacencyListGraph
blockingFlow(LevelGraph& level_graph, vertex_t src, vertex_t dest);


/**
 * algorithme de Dinic
 * @param flow_network le réseau de transport
 * @param src le sommet source
 * @param dest le puit
 * @return le graphe d'écart final
 */
AdjacencyListGraph
dinic(const AbstractGraph& graph, vertex_t src, vertex_t dest);

\end{verbatim}