\subsection{SAT $\propto$ 3--SAT}
\begin{enumerate}[(a)]
\item \begin{description}
\item[Énoncé de SAT] : \\
\begin{tabular}{r l l}
Données : & $ \mathcal{V} = \lbrace v_1, v_2 \ldots v_n \rbrace $ & \emph{Ensemble de $n$ variables}\\
& $ \mathcal{C} = \lbrace c_1, c_2, c_3 \ldots c_m \rbrace $ & \emph{Ensemble de $m$ clauses}\\
où & $ c_i = ( l_{i1} \vee l_{i2} \vee \cdots \vee l_{ik} ) $ & \emph{Clauses de $k$ littéraux}\\
avec & $ l_{ij} = v$ ou $ \neg v $ & \emph{avec $v \in U$} \\
\end{tabular}

Problème : existe-il au moins une affectation des variables telle que chaque clause de $\mathcal{C}$ soit vrai.

\item [Énoncé de 3--SAT] : 

3--SAT est identique au problème SAT avec $k = 3$.\\
\begin{tabular}{r l}
Données : & $ \mathcal{V} = \lbrace v_1, v_2, v_3 \ldots v_n \rbrace $\\
& $ \mathcal{C} = \lbrace c_1, c_2, c_3 \ldots c_m \rbrace $\\
où & $ c_i = ( l_{i1} \vee l_{i2} \vee l_{i3} ) $\\
avec & $ l_{ij} = v$ ou $ \neg v$\\
\end{tabular}
\end{description}
\item La réduction du problème SAT peut être définit en montrant que chaque clause $c$ de $\mathcal{C}$ peut-être transformée en un ensemble de clauses $\mathcal{C'}$ tel que pour toute affectation rendant vrai l'ensemble des clauses de $\mathcal{C}$, on peut trouver une affectation rendant vrai chaque clause de $\mathcal{C'}$. Chaque clause de $\mathcal{C'}$ devant être de taille exactement 3. La réciproque doit également être montrée.

Définissons les réductions :

\begin{description}
\item \mathversion{bold} $k = 1$ \mathversion{normal}

Soit $ci_1$ une clause de taille 1, on a $ci_1 = (l)$.
Ajoutons deux variables $v_1, v_2 \notin \mathcal{V}$ et transformons la clause $c$ en quatre clauses. On obtient l'ensemble $\mathcal{C}_1 = \lbrace c_1, c_2, c_3, c_4 \rbrace$ avec :
\[ c_1 = ( l \vee v_1 \vee v_2 )\]
\[ c_2 = ( l \vee v_1 \vee \neg v_2 )\]
\[ c_3 = ( l \vee \neg v_1 \vee v_2 )\]
\[ c_4 = ( l \vee \neg v_1 \vee \neg v_2 )\]

\item \mathversion{bold} $k = 2$ \mathversion{normal}

Soit $ci_2$ une clause de taille 2, on a $ci_2 = ( l_1 \vee l_2 )$.
Ajoutons une variable $v \notin \mathcal{V}$ et transformons la clause $c$ en deux clauses. On obtient l'ensemble $\mathcal{C}_2 = \lbrace c_1, c_2 \rbrace$ avec :
\[ c_1 = ( l_1 \vee l_2 \vee v )\]
\[ c_2 = ( l_1 \vee l_2 \vee \neg v )\]

\item \mathversion{bold} $k = 3$ \mathversion{normal}

La clause $ci_3$ ne subit pas de transformation.
\[ \mathcal{C}_3 = \lbrace ci_3 \rbrace \]

\item \mathversion{bold} $k > 3$ \mathversion{normal}

Soit la clause $ci_k = ( l_1 \vee l_2 \vee \cdots \vee l_k )$. On ajoute $(k - 3)$ nouvelles variables $(v_1, v_2 \ldots v_{k-3})$.

\[ \mathcal{C}_k = \underbrace{(l_1 \vee l_2 \vee v_1)}_{c_1} \bigwedge_{i=1}^{k-4}\left[ \underbrace{(\neg v_i \vee l_{i+2} \vee v_{i+1})}_{c_{i+1}}\right]  \wedge \underbrace{(\neg v_{k-3} \vee l_{k-1} \vee l_{k})}_{c_{k-2}} \]

Montrons que SAT est vrai si et seulement si 3--SAT est vrai :

\begin{description}
 \item \textbf{SAT $\rightarrow$ 3--SAT}
 
 	\begin{itemize}
 		\item Soit une interprétation $I_1$ qui satisfasse la clause $ci_1$ :
 		 \[ val(I_1,ci_1) = val(I_1,l) = vrai\]
 		
 		Prenons une interprétation $I_1'$ avec $val(I_1,l) = val(I_1',l)$, peu importe les affectations de $v_1$ et $v_2$, $l$ étant présent dans toutes les clauses de $\mathcal{C}_1$ :
 		\[ val(I_1',\mathcal{C}) = vrai \]
 		
 		\item Soit une interprétation $I_2$ qui satisfasse la clause $ci_2$ :
 		\[ \exists i, val(I_2,l_i) = vrai \]
 		Prenons une interprétation $I_2'$ avec :
 		\[ val(I_2,l_1) = val(I_2',l_1) \]
  		\[ val(I_2,l_2) = val(I_2',l_2) \]
  		Peu importe l'affectation de $v$ dans $I_2'$, on a $val(I_2',\mathcal{C}_2) = vrai$.
  		
  		\item Soit une interprétation $I_k$ qui satisfasse la clause $ci_k$ :
  		\[ \exists i, val(I_k,l_i) = vrai \]
  		
  		Prenons une interprétation $I_k'$ telle que :
  		\begin{eqnarray*}
  		val(I_k,l_i) & = & val(I_k',l_i)  \\
  		\forall j \in [1;(i-2)], val(I_k', v_j) & = & vrai \\
  		\forall j \in [(i-1);(k-3)], val(I_k', v_j) & = & faux \\
  		\end{eqnarray*}
  		
  		On obtient :
  		\[ val(I_k',\mathcal{C}_k) = vrai\]
  		
 	\end{itemize}
 	
 \item \textbf{3--SAT $\rightarrow$ SAT}
 
 	\begin{itemize}
 	\item Prenons une interprétation $I_1$ telle que $val(I_1,\mathcal{C}_1) = vrai$.
 	
 	Sans perte de généralité, on suppose que : 
 	\[ val(I_1,v_1) = val(I_1,v_2) = vrai \]
 	La clause $c_4$ de $\mathcal{C}_1$ ne peut être satisfaite que si $val(I_1,l) = vrai$.
 	
 	On a donc :
 	\[ val(I_1,ci_1) = vrai \]
 	
 	\item Prenons une interprétation $I_2$ telle que $val(I_2,\mathcal{C}_2) = vrai$.
 	
 	Sans perte de généralité on suppose que :
 	\[val(I_2,v) = vrai \]
 	
 	La clause $c_2$ de $\mathcal{C}_2$ ne peut être satisfaire que si $val(I_2,(l_1 \vee l_2)) = vrai$.
 	
 	On a donc :
 	\[ val(I_2,ci_2) = vrai \]
 	
 	\item Prenons une interprétation $I_k$ telle que $val(I_k,\mathcal{C}_k) = vrai$ et montrons qu'il existe forcément un $i$ tel que $val(I_k,l_i) = vrai$.
 	
 	Supposons que l'interprétation $I_k$ est modèle de $\mathcal{C}_k$ avec 
 	\[ \forall i \in [1;k], val(I_k,l_i) = faux \]
	\[ \Rightarrow val(I_k,v_1) = vrai  \textrm{ (dans $c_1$)} \]
	Donc :
	
	\begin{tabular}{lrl}
		& $\forall i \in [1;(k-4)], val(I_k,v_{i+1})$ & $= vrai$ \\
		$\Rightarrow$ & $val(I_k,v_{k-3})$ & $= vrai$ \\
		$\Rightarrow$ & $val(I_k,c_{k-2})$ & $= faux$ \\
		$\Rightarrow$ & $val(I_k,\mathcal{C}_k)$ & $= faux$
	\end{tabular}
	
	Pour que l'interprétation $I_k$ satisfasse $\mathcal{C}_k$, il doit exister un $i \in [1;k]$ tel que $val(I_k,l_i) = vrai$.
	
	On a donc :
	\[ val(I_k,ci_k) = vrai \]
 	 	
 	\end{itemize}
 
\end{description}

\end{description}

\item Le point (b) définit la réduction de SAT vers 3--SAT. Afin de montrer la NP-Complétude de 3--SAT, montrons que la réduction s’effectue en un temps polynomial.

Soit :
\begin{description}
\item $k$ la taille de la clause initiale,
\item $v_k$ le nombre de variables à ajouter pour obtenir des clauses de taille 3,
\item $w_k$ le nombre de clauses de taille 3 obtenues à partir de la clause initiale.
\end{description}
\begin{center}
\begin{tabular}{c c}
$v_3 = 0$ & $w_3 = 1$ \\
$v_4 = 1$ & $w_4 = 2$ \\
$v_5 = 2$ & $w_5 = 3$ \\
\vdots & \vdots
\end{tabular}
\end{center}

Pour tout $k > 3$ :
\[ v_k = v_{\left \lceil \frac{k}{2} \right \rceil + 1} + v_{\left \lfloor \frac{k}{2} \right \rfloor + 1} + 1 \]
\[ w_k = w_{\left \lceil \frac{k}{2} \right \rceil + 1} + w_{\left \lfloor \frac{k}{2} \right \rfloor + 1} \]

$v_k = \theta(k)$, donc borné par la taille de F. La réduction s'effectue donc en un temps polynomial.

Il est possible de réduire le problème SAT à 3--SAT en un temps polynomial, SAT étant NP-complet, 3--SAT l'est aussi.
\item Soit $\mathcal{C}$ un ensemble de clause à $n_v$ variables avec $n_1$ clauses de taille 1, $n_2$ clauses de taille 2, $n_3$ clauses de taille 3, $n_4$ clauses de taille 4 et $n_5$ clauses de taille 5. Calculons le nombre de variables et le nombre de clauses obtenues après réduction (respectivement $n_v'$ et $n_c'$).

Les points (b) et (c) permettent de déterminer pour une clause de taille $k$, le nombre de clause obtenues et le nombre de variables ajoutées après réduction. On peut donc en déduire la tableau suivant :

\begin{tabularx}{\textwidth}{| X || c | c | c | c | c |}
\hline
Taille de la clause dans $\mathcal{C}$	& 1 	& 2 	& 3 	& 4 	& 5 	\\
\hline
Nombre de clauses						& $n_1$	& $n_2$	& $n_3$	& $n_4$	& $n_5$	\\
\hline
Nombre de variables ajoutées par clause	& 2		& 1 	& 0 	& 1 	& 2 	\\
\hline
\textbf{Nombre de variables ajoutées au total} 	& $2n_1$& $n_2$	& 0		& $n_4$	& $2n_5$\\
\hline
Nombre de clauses obtenues par clause 	& 4 	& 2 	& 1 	& 2 	& 3 	\\
\hline
\textbf{Nombre de clauses obtenues au total}	& $4n_1$& $2n_2$& $n_3$	& $2n_4$& $3n_5$\\
\hline

\end{tabularx}

On a donc :
\[ n_v' = n_v + 2n_1 + n_2 + n_4 + 2n_5 \]
\[ n_c' = 4n_1 + 2n_2 + n_3 + 2n_4 + 3n_5 \]
\end{enumerate}

\subsection{3--SAT $\propto$ 2--SAT ?}
Cette réduction repose sur un principe qui consiste à décomposer une clause de taille $k$ en plusieurs clauses de tailles inférieures.

Soit une clause $c = (l_1 \vee l_2 \vee l_3)$ une clause de taille 3 et $I$ une interprétation qui satisfait $c$.

\begin{description}
\item[Cas 1 :] décomposons cette clause en deux clauses $c_1$ et $c_2$ de tailles 1 et 2 :
\begin{eqnarray*}
c_1&=&(l_1) \\
c_2&=&(l_2 \vee l_3)
\end{eqnarray*}
Pour montrer l'équivalence 3--SAT $\leftrightarrow$ 2--SAT, il faut ajouter une variable $v$ aux deux clauses créées :
\begin{eqnarray*}
c_1&=&(l_1 \vee v) \\
c_2&=&(l_2 \vee l_3 \vee \neg v)
\end{eqnarray*}

On a donc la clause $c_2$ de taille 3.

\item[Cas 2 :] décomposons cette clause en trois clauses $c_1$, $c_2$ et $c_3$ de taille 1 :
\begin{eqnarray*}
c_1 & = & (l_1) \\
c_2 & = & (l_2) \\
c_3 & = & (l_3)
\end{eqnarray*}

Pour montrer l'équivalence 3--SAT $\leftrightarrow$ 2--SAT, il faut ajouter deux variables $v_1$ et $v_2$ aux trois clauses créées :
\begin{eqnarray*}
c_1 & = & (l_1 \vee v_1 \vee \neg v_2) \\
c_2 & = & (l_2 \vee \neg v_1 \vee v_2)\\
c_3 & = & (l_3 \vee v_1 \vee v_2)
\end{eqnarray*}
On a donc également des clauses de taille 3. La réduction définie ci-avant ne permet donc pas la réduction de 3--SAT vers 2--SAT.
\end{description}




\subsection{2--SAT, un problème polynomial}
\begin{enumerate}
\item coucou
\item coucou
\end{enumerate}
