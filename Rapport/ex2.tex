\subsection{Réduction de SAT à 3--SAT}
\begin{enumerate}[(a)]
\item \begin{description}
\item[Énoncé de SAT] : \\
\begin{tabular}{r l l}
Données : & $ U = \lbrace v_1, v_2 \ldots v_n \rbrace $ & \emph{Ensemble de $n$ variables}\\
& $ F = \lbrace c_1, c_2, c_3 \ldots c_m \rbrace $ & \emph{Ensemble de $m$ clauses}\\
où & $ C_i = l_{i1} \vee l_{i2} \cdots l_{ik} $ & \emph{Clauses de $k$ littéraux}\\
avec & $ l_{ij} = v$ ou $ \neg v $ & \emph{avec $v \in U$} \\
\end{tabular}

Problème : existe-il au moins une affectation des variables telle que chaque clause de $F$ soit vrai.

\item [Énoncé de 3--SAT] : 

3--SAT est identique au problème SAT avec $k = 3$.\\
\begin{tabular}{r l}
Données : & $ U = \lbrace v_1, v_2, v_3 \ldots v_n \rbrace $\\
& $ F = \lbrace c_1, c_2, c_3 \ldots c_m \rbrace $\\
où & $ C_i = l_{i1} \vee l_{i2} \vee l_{i3} $\\
avec & $ l_{ij} = v$ ou $ \neg v$\\
\end{tabular}
\end{description}
\item La réduction du problème SAT peut être définit en montrant que chaque clause $C$ de $F$ peut-être transformée en un ensemble de clauses $\lbrace c_1, c_2, \ldots, c_i \rbrace$ tel que pour toute affectation rendant vraie $C$, on peut trouver une affectation rendant vrai chaque $c_i$. Chaque $c_i$ devant être de taille exactement 3. La réciproque doit également être montrée.

\begin{description}
\item \mathversion{bold} $k = 4$ \mathversion{normal}

Soit $C$ une clause de taille 4 : $C = l_1 \vee l_2 \vee l_3 \vee l_4$.
Ajoutons une variable $v \notin U$ et transformons la clause $C$ en deux clauses :
\[ C_1 = l_1 \vee l_2 \vee v \]
\[ C_2 = l_3 \vee l_4 \vee \neg v \]
\begin{description}
\item [4 $\Rightarrow$ 3]
Soit une affectation qui rend $C$ vrai. Il existe un $i$ tel que $l_i$ est vrai.

Sans perte de généralité, prenons $i = 1$. Conservons les mêmes affectations pour les $l_i$ avec $v = faux$.

On a $C_1 \wedge C_2 = vrai$.
\item [3 $\Rightarrow$ 4]
Soit une affectation telle que $C_1 \wedge C_2$ est vrai.

Sans perte de généralité, on suppose que $v$ est vrai. Cela implique que $l_3 \vee l_4$ est vrai, et donc que $C$ est vrai.
\end{description}

\item \mathversion{bold} $k = 5$ \mathversion{normal}

Soit la clause $C = l_1 \vee l_2 \vee l_3 \vee l_4 \vee l_5$, sur le même principe que pour $k = 4$ :
\[C \rightarrow (l_1 \vee l_2 \vee l_3 \vee v) \wedge (l_4 \vee l_5 \vee \neg v) \]
\[\rightarrow (l_1 \vee l_2 \vee w) \wedge (l_3 \vee v \vee \neg w) \wedge (l_4 \vee l_5 \vee \neg v)\]

La clause $C$ est transformée en 3 clauses de taille 3, en ajoutant 2 nouvelles variables.

\item [Généralisation] :

Une clause de taille $k > 3$ est transformée en 2 clauses $C_1$ et $C_2$ en ajoutant une variable :
\begin{itemize}
\item Taille de $C_1$ : $\left \lceil \frac{k}{2} \right \rceil + 1$
\item Taille de $C_2$ : $\left \lfloor \frac{k}{2} \right \rfloor + 1$
\end{itemize}
Avec $k > 3$, chaque réduction diminue stictement la taille des clauses, car on a : \[ k > \left \lceil \frac{k}{2} \right \rceil + 1 \geq \left \lfloor \frac{k}{2} \right \rfloor + 1 \]

Tout problème SAT peut dont être réduit à un problème 3--SAT.

\end{description}

\item Le point (b) définit la réduction de SAT à 3--SAT.

Soit :
\begin{description}
\item $k$ la taille de la clause initiale,
\item $v_k$ le nombre de variables à ajouter pour obtenir des clauses de taille 3,
\item $w_k$ le nombre de clauses de taille 3 obtenues à partir de la clause initiale.
\end{description}
\begin{center}
\begin{tabular}{c c}
$v_3 = 0$ & $w_3 = 1$ \\
$v_4 = 1$ & $w_4 = 2$ \\
$v_5 = 2$ & $w_5 = 3$ \\
\vdots & \vdots
\end{tabular}
\end{center}

Pour tout $k > 3$ :
\[ v_k = v_{\left \lceil \frac{k}{2} \right \rceil + 1} + v_{\left \lfloor \frac{k}{2} \right \rfloor + 1} + 1 \]
\[ w_k = w_{\left \lceil \frac{k}{2} \right \rceil + 1} + w_{\left \lfloor \frac{k}{2} \right \rfloor + 1} + 1 \]
\end{enumerate}

\subsection{NP-complétude de 2--SAT}
\begin{enumerate}
\item coucou
\item coucou
\end{enumerate}

\subsection{2-SAT, un problème polynomial}
\begin{enumerate}
\item coucou
\item coucou
\end{enumerate}
